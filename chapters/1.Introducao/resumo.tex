\begin{resumo}
	
	A computação ubíqua tem por objetivo a criação de ambientes inteligentes, conhecidos também 
	por~\textit{smart spaces}. O objetivo destes é possibilitar uma interação pró-ativa para o ambiente 
	com seus usuários. Esta pró-atividade deve ser realizada da maneira transparente e minimizando 
	a intrusão nas atividades sendo realizadas. Porém, para que isto ocorra, é necessária a análise 
	das informações disponíveis para que se conheça o contexto do usuário. Estas informações são 
	obtidas através dos dispositivos presentes no ambiente e são necessárias para fornecer a 
	inteligência almejada. Camadas de software denominadas~\textit{middlewares} são utilizadas 
	para obter e tratar estas informações delegando as ações as aplicações construídas sobre elas. 
	O~\textit{uOS} é um exemplo de~\textit{middleware} focado em permitir o acesso aos dispositivos 
	do ambiente.

	A \textit{Hydra} é uma aplicação construída utilizando o~\textit{uOS} que possibilita ao usuário 
	redireciona seus recursos para outros mais adequados no ambiente. Esta aplicação visa facilitar 
	o usuário em suas atividades fornecendo uma maior gama de opções ao realizá-las. Para isso, a 
	aplicação foca em apresentar de forma textual os dispositivos e recursos disponíveis. No entanto, 
	a tarefa de se localizar estes no ambiente aumenta com a quantidade de dispositivos presentes. 
	A fim de minimizar esta tarefa ao usuário, esse trabalho apresenta uma aplicação denominada de 
	ARHydra (\textit{Augmented Reality Hydra}). 
	
	Esta objetiva em prover uma interface de interação aprimorada 
	à Hydra, permitindo ao usuário uma maior transparência e facilidade na escolha dos recursos do 
	ambiente.  Para isso é feito o uso das técnicas de Realidade Aumentada, permitindo uma integração 
	entre as informações e o ambiente real.	
	A ARHydra faz uso de marcadores, utilizando o QRCode em sua composição, para mapear os 
	dispositivos dentro do ambiente inteligente e a partir deste, obter e apresentar os recursos 
	por ele disponibilizados. Para medir a influência dos dispositivos e do ambiente na execução da 
	ARHydra, serão apresentados	os resultados obtidos nos testes efetuados mostrando um comparativo 
	de desempenho da ARHydra entre dois~\textit{smartphones} e a influência de aspectos 
	envolvidos no processo de captura da imagem e na composição do QRCode que possam interferir nestes 
	resultados. 
	 
\end{resumo}

\selectlanguage{american}
	
\begin{abstract}

	Ubiquitous computing aims at creating smart environments, also known by smart spaces. The goal of these 
	interactions is enable a proactive environment for its users. This pro-activity should be done in 
	a transparent manner and minimizing the intrusion on the activities being performed. However, for this 
	to occur, is necessary to analyse of available information so that know the context of the user. 
	These informations are obtained through the devices in the environment and are necessary to provide the desired 
	intelligence. Layers of software called middleware are used to obtain information and treat these actions 
	delegating applications built on them. The uOS is an example of middleware focused on allowing access to 
	devices in the environment.
	
	The Hydra is an application built using the uOS that allows the user redirect its resources to other more 
	suitable environment. This application aims to facilitate the user in their activities by providing a 
	greater range of options to perform them. For this, the application focuses on presenting in textual 
	devices and resources availiable. However, the task of locating these environmental increases with the 
	number of devices present. In order to minimize this task to the user, this work presents an application 
	called ARHydra (Augmented Reality Hydra). 
	
	This aims to provide an improved interface for the interaction with the 
	Hydra, allowing greater transparency to the user and ease in choosing the environment resources. For this 
	are ​​used the techniques of Augmented Reality, allowing integration between the information and the 
	real environment. The ARHydra uses markers, using QRCode in their composition, to map the devices within 
	the smart space and from this, obtain and provide the resources has made available. To measure the influence of 
	the devices and the environment in the execution of ARHydra, will present the results of those tests showing a 
	comparative	performance of the  ARHydra between two smartphones and the influence of aspects involved 
	in image capture and in the QRCode composition that may interfere these results.


\end{abstract}