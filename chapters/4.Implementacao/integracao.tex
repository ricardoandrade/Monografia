\section{Módulo de Integração} 
\label{sec:modulo_integracao}

	Na composição do objeto virtual é necessário reunir informações relevantes sobre a disponibilidade dos 
	recursos provido pelo dispositivo escolhido. O Módulo de Integração é responsável pela integração da 
	ARHydra com a Hydra com o propósito de criar um canal de comunicação entre essas duas aplicações. Esta 
	integração é feita através de \textit{drivers} de comunicação, conforme	estabelecido pela DSOA. 
	
	Quando um recurso apresentado pelo dispositivo é selecionado faz-se necessário informar a Hydra	qual 
	recurso fora escolhido. Esse canal externo de comunicação entre as aplicações não é fornecida pelo 
	\textit{uOS} de forma nativa. Por essa razão, a arquitetura inicial da Hydra 
	não implementou nenhuma forma para o recebimento de requisições externas. Permite apenas a interação 
	via terminal de aplicação, ou seja, sem acesso por outras aplicações. Desta forma, para o usuário 
	redirecionar algum recurso, ele terá que ir até o dispositivo executando a Hydra e prover o 
	redirecionamento de forma manual. Para realizar essa comunicação foi desenvolvido um~\textit{driver} 
	na Hydra que possibilitasse o recebimento de requisições externas. O envio das requisições, bem como
	o recebimento das respostas realizadas pelas mesmas, decorrentes dessa comunicação, ocorre através do 
	Módulo de Integração.  

	Para economizar recursos do \textit{smartphone}, a obtenção da informação de quais dispositivos
	estão disponíveis dentro do ambiente inteligente fica sob responsabilidade da Hydra, ao qual implementará
	mecanismos para identificação dos dispositivos no ambiente através da utilização de radares. A ARHydra 
	possui um mecanismo de agendamento para que essa informação seja atualizada. Com a periodicidade de 60 segundos, 
	o Módulo de Integração envia informações para a Hydra requisitando quais são os dispositivos que estão 
	ativos dentro do ambiente.
	