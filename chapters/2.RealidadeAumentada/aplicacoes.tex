\section{Aplicações}
\label{sec:aplicacoes}

	Tendo em vista o potencial de aplicação da Realidade Aumentada na interação com usuários muitos estudos tem
	sido realizados neste sentido. Aplicações tem sido desenvolvidas em diversos ramos da atividade humana. 
	A seguir exemplifica-se alguns usos da Realidade Aumentada:
	
	\begin{itemize}
	  \item \textbf{Entretenimento} 
	  
	  		É nessa área que encontra-se o maior número de aplicações utilizando a Realidade
	  		Aumentada, sendo estas desenvolvidas principalmente para a área de jogos. O grande
	  		diferencial em utilizar a Realidade Aumentada em jogos é fazer com que seus cenários interajam
	  		com o mundo real ao qual o usuário está inserido. Isso possibilita uma interação maior
	  		do usuário e o cenário apresentado pelos jogos. 
	  		
	  \item \textbf{Medicina} 
	  
	  		O uso da Realidade Aumentada vem auxiliando a medicina em muitos aspectos, desde a visualização
	  		partes do corpo até a sua utilização em cirurgias. O \textit{HMD (Head Mounted Display)} é um
	  		equipamento bastante utilizado para auxiliar na visualização dos objetos virtuais,
	  		possibilitando que o mesmo seja seja utilizado em cirurgias guiadas por imagens. Deste modo,
	  		essa aplicação da Realidade Aumentada auxiliará em um melhor planejamento cirúrgico,
	  		contribuindo para uma diminuição dos riscos envolvidos~\cite{suthau, nilsson}.
	
	  \item \textbf{Mercado imobiliário e arquitetura} 
	  
	  		Maquetes e \textit{design} de interiores são construídas utilizando Realidade Aumentada com
	  		propósito de exemplificar aos consumidores o estado do empreendimento ao término de sua
	  		construção. Desta forma, a Realidade Aumentada é bastante utilizada nessas áreas para
	  		possibilitar aos compradores visualizar e customizar seus ambientes, possibilitando a
	  		modificação da disposição dos objetos e a observação dessas novas disposições.
	  		%com que o comprador faça modificações nas disposições dos objetos do jeito que achar
	  		% necessário e observar suas novas disposições.

	  \item \textbf{Auxílio na obtenção de informações a respeito de produtos} 
	  
	  		Empresas utilizam a Realidade Aumentada com o propósito de oferecer maiores detalhes a respeito
	  		de seus produtos. Eles são identificados por marcadores reconhecidos pela Realidade Aumentada.
	  		As informações necessárias são extraídas após o reconhecimento desse marcador e um objeto
	  		virtual contendo informações a respeito do produto é apresentado ao usuário. No caso de uma
	  		rede de supermercados, tais informações poderiam representar descontos, opiniões e ingredientes
	  		de produtos anunciados. Utilizando ainda o exemplo anterior, a localização dos produtos dentro
	  		do estabelecimento poderia ser mapeada de uma forma com que uma aplicação pudesse auxiliar o
	  		usuário a encontrar um determinado produto, através de uma navegação guiada por GPS
	  		ou por outros marcadores que guardariam a localização referentes a cada tipo de produto.
			
							  		
	  		%A visualização 3D do conteúdo interior de um produto a partir de marcadores posicionados nas
	  		%caixas do produto, proporcionou empresas utilizarem a realidade aumentada para possibilitar
	  		% com que os clientes possam obter maiores detalhes sobre o produto antes da compra.
	  
	  %%TODO arrumar a referência Augmented Reality Technology for Education, Mariano Alcatriz
	  % (procurar o bibtex)
	  \item \textbf{Educação}  
	  
	  		Através dos benefícios providos pela flexibilidade e usabilidade, a Realidade
	  		Aumentada foi utilizada na educação com o foco na aprendizagem. Sua utilização vai
	  		além da aprendizagem utilizando somente os livros, ela explora características que até então
	  		não eram percebidas no ambiente acadêmico, potencializando sua aprendizagem devido
	  		principalmente a interatividade provida pela Realidade Aumentada
	  		\cite{kaufmann,markBillinghurst}. Para exemplificar essa aplicação na Realidade Aumentada, 
	  		objetos dentro do museu britânico foram mapeados utilizando marcadores reconhecidos pela
	  		Realidade Aumentada proporcionando aos visitantes obterem informações a respeito dos objetos
	  		apresentados~\cite{museum}. Por outro lado, a Realidade Virtual vem sendo aplicada na educação
	  		desde a década de 90. Esses projetos proporcionam a criação de um mundo virtual com o propósito
	  		de ensinar e investigar os aspectos relacionados, como por exemplo, da cinemática,
	  		eletrostática, estruturas moleculares, estudo da biologia e matemática~\cite{ko, hannes}.
			
	  %% ~\cite{mariano}
	
	  \item \textbf{Turismo} 
	  
	  		Essa área tem sido bastante explorada devido a possibilidade de mapeamento de pontos
	  		turísticos e disponibilização de informações através desse mapeamento. Essas informações podem
	  		ser disponibilizadas de acordo com o perfil do usuário, com objetivo de auxiliá-lo na busca de
	  		locais de seu interesse. A localização desses pontos é feito através das coordenadas de
	  		localização do usuário, obtidas através de um GPS. Essas informações são cruzadas com posições
	  		de longitude, latitude e altitude obtidas de um banco de dados contendo todos os mapeamentos
	  		feitos. Na Realidade Aumentada essas posições mapeadas são denominadas de POI's (\textit{Points Of
	  		Interest}). Por causa da mobilidade obtida por esses recursos, eles são encontrados
	  		principalmente em aplicações voltadas para dispositivos móveis.
	
	\end{itemize}
	
	
	Como observado, a Realidade Aumentada é utilizada em diversas áreas com base em uma característica
	comum entre elas, a possibilidade de interação entre o real e o virtual. A visualização dos recursos
	providos pela Realidade Aumentada necessita de equipamentos compatíveis. Esses equipamentos
	proporcionam a captura de imagens, correspondente ao ambiente real do usuário, a construção e
	apresentação de objetos virtuais ao usuário.
	
