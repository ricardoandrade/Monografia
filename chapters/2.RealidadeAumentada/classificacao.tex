\section{Classificação}
\label{sec:classificacao}
	
	Com a evolução dos equipamentos, a visualização dos objetos virtuais torna-se cada vez mais real.
	Através desses equipamentos, os sistemas de Realidade Aumentada podem ser classificados de acordo
	com a forma como que o usuário vê o mundo misturado. Em~\cite{ronaldAzuma} é apresentado uma
	divisão para classificações entre as tecnologias óptica e vídeo. Essas classificações variam de
	acordo com o tipo de equipamento utilizado e com o tipo de sistema de visualização:
	
	\begin{enumerate}
		\item \textbf{Visão direta}
		
		Também conhecida como visão imersiva, nesta classificação os objetos virtuais são visualizados
		na mesma direção com que as cenas reais são capturadas. Esta classificação é utilizada
		principalmente em equipamentos que capturam as imagens reais, processam as informações
		necessárias e apresentam objetos ao usuário no mesmo equipamento~\cite{suthau}. Por
		exemplo, a utilização de equipamentos do tipo \textit{HMD} proporciona ao usuário uma visão
		direta, tendo em vista que a captura das cenas e a apresentação do objeto virtual foi feito pelo
		mesmo equipamento. Desta forma o usuário não necessita desviar o foco do mundo real a fim de
		observar as informações providas pelo objeto virtual.
		
		%Isso proporciona com que o usuário não precise desviar o olhar para a visualização do objeto
		% virtual.
		
		\item \textbf{Visão indireta}
		
		Ocorre quando a visualização do mundo misturado é feita através de algum dispositivo e a
		apresentação dos objetos virtuais, correspondentes as cenas, seja feita em um outro dispositivo,
		ocasionando o desvio da atenção do usuário~\cite{suthau, kernerTori}. Um exemplo para
		esse tipo de classificação é observado quando a captura das imagens reais é feita através de uma
		\textit{webcam}, processadas em um computador e o resultado apresentado em projeções ou em
		monitores.
		
	\end{enumerate}
	
	Uma outra forma de se classificar a Realidade Aumentada baseia-se nos sistemas utilizados por ela.
	Estes podem ser classificados de acordo com o tipo de equipamento utilizado para captura de cenas
	reais, processamento e os equipamentos responsáveis pela visualização do objeto virtual. Estas
	classificações abrangem tanto sistemas que utilizem visão óptica quanto visão por vídeo. Sendo
	classificados em:
		
	\begin{enumerate}
	
		\item \textbf{Visão direta por vídeo} 
		
		Neste tipo de classificação, equipamentos \textit{HMD} são utilizados para o recebimento direto
		da imagem real através de suas câmeras acopladas. As imagens capturadas são processadas em
		um gerador de cenas e as informações virtuais geradas por este são apresentadas diretamente ao
		usuário, através das telas acopladas ao equipamento. 
		
		%Nesta, um equipamento é utilizado para combinar cenas reais com as virtuais, a partir de
		%imagens obtidas através desse equipamento composto por pequenas câmeras de vídeo. As imagens
		%capturadas são processadas e novas informações virtuais são geradas por um computador. Essas
		%informações são misturadas com as cenas reais capturadas pelas micro câmeras acopladas ao
		%equipamento e as novas informações geradas pelo gerador de cenas são mostradas diretamente ao
		%usuário através das telas acopladas ao equipamento.
		
		\item \textbf{Visão direta por óptica} 
		
		De forma análoga a visão direta por vídeo, esta possui as mesmas etapas para obtenção e geração
		as informações virtuais a serem apresentadas ao usuário. No entanto, as informações geradas são
		apresentadas ao usuário através de lentes ou espelhos. Estas são inclinadas e posicionadas
		para que a projeção das imagens vindas do monitor acoplado seja refletida e redirecionada para
		os olhos do usuário.
		
		\item \textbf{Visão por vídeo baseada em monitor} 
		
		Após a captura das cenas, os objetos virtuais são gerados no gerador de cenas e misturados com
		as cenas reais no combinador de cenas. As informações geradas são apresentadas ao usuário
		através de um monitor.
		
		\item \textbf{Visão por óptica baseada em projeção} 
		
		Nessa classificação, as imagens dos objetos virtuais são apresentadas sem a necessidade de um
		equipamento específico. As mesmas são processadas em computador e projetadas em superfícies do
		ambiente real, por causa da necessidade de superfícies específicas para projeções, essa
		classificação torna-se mais restrita em comparação com as demais classificações.
		
	\end{enumerate}
