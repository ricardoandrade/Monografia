\subsection{~\textit{uP}}
\label{sec:up}


	Devido a grande diversidade e níveis de processamentos de dispositivos presentes no ~\textit{smart
	space}, tornou-se necessário a busca de uma maneira mais simples e padronizada para que os mesmos
	pudessem trocar as informações. Buzeto ~\cite{uos},	propõe um protocolo para a computação ubíqua,
	implementada sobre o ~\textit{middleware uOS}, denominada de ~\textit{Ubiquitous Protocols (uP)}.
	Esse conjunto de protocolos permite uma interação entre os dispositivos levando-se em conta a
	heterogenidade das plataformas, os diferentes modelos de comunicação e interação com o ambiente.
	
	O protocolo ~\textit{uP} utiliza a tecnologia ~\textit{JavaScript Object Notation (JSON)} para o
	tráfego de dados entre os dispositivos. Tanto tecnologia ~\textit{JSON} quanto o
	~\textit{Extensible Markup Language (XML)}, possuem os conceitos herdados do ~\textit{SOA}, motivo
	que viabilizou a sua utilização na arquitetura ~\textit{DSOA}. Algumas características são
	similares nessas tecnologias, tais como:  independência de plataforma, formato estruturado e
	auto-descritivo.
	
	Podemos citar algumas vantagens da tecnologia ~\textit{JSON}, quando comparada ao ~\textit{XML}
	~\cite{enokiFreitas}, que viabilizaram seu uso no ~\textit{uP}:

		\begin{enumerate}
		  \item baixo custo computacional
          \item carga de processamento
          \item transferência dos dados
          \item demandam menor esforço
          \item economia de recursos computacionais e bateria, devido a redução no tempo de
          tratamento das mensagens.
		\end{enumerate}
 
	
	% --------------------- INTERFACES ---------------------
	
	\subsubsection{Interfaces}
	
	Através de interfaces públicas pré definidas, o ~\textit{uP} implementa os conceitos apresentados
	pela ~\textit{DSOA} da seguinte maneira:
	
	\begin{enumerate}
	  \item \textit{UpDevice}

			Representa, de forma única, os dispositivos presentes no ambiente inteligente. Possui as
			propriedades ~\textit{Name}, responsável por identificar o dispositivo, e  ~\textit{Networks},
			representando uma lista de redes de comunicação presentes no dispositivo, onde cada elemento da
			lista é composto pelo tipo de conexão da rede e seu endereço.
	  
	  \item \textit{UpDriver}
	  		
	  		Esta por sua vez, define as propriedades dos recursos do dispositivo. Cada instância possui um
	  		identificador ~\textit{InstanceID} porque o mesmo ~\textit{driver} pode ser instanciado mais de
	  		uma vez.
			
			Possui as propriedades ~\textit{Name}, responsável por identificar o recurso, ~\textit{Services}
			e ~\textit{Events}, listas de serviços síncronos e assíncronos, respectivamente.
			  
	  \item \textit{UpService}
			
			Representa o conceito de serviço. Também possui um atributo ~\textit{Name}, responsável por
			identificar o serviço, e ~\textit{Parameters}, informando os parâmetros necessários para a
			execução, sendo estes obrigatórios ou opcionais.
	
	\end{enumerate}
	
	% --------------------- TIPOS DE MENSAGENS ---------------------
	
	\subsubsection{Tipos de mensagens}
	
	O ~\textit{uP} implementa quatro tipos de mensagens possibilitando assim uma comunicação e
	interação entre os dispositivos no ambiente. São elas:
	
	\begin{enumerate}
	  	\item \textit{Service Call}
		
			Troca de mensagens em serviços síncronos. Aceita parâmetros opcionais e obrigatórios definidos
			em suas propriedades.
		
		\item \textit{Service Response}
			
			Responsável por retornar uma mensagem contendo informações de resposta a solicitação do
			~\textit{Service Call}.
		
		\item \textit{Notify}
		
			Resposta a uma requisição assíncrona de um evento.
	
		\item \textit{Encapsulated Message}
		
			São utilizadas para encapsular mensagens dentro de outras mensagens, permite uma codificação de
			mensagens. Essa encriptação favorece o estabelecimento de canais de comunicação mais seguros.
	
	\end{enumerate} 
	
	% --------------------- PROTOCOLOS BÁSICOS --------------------
	
	
	\subsubsection{Protocolos}
	
	
	
	
	
	
	
	
	
	