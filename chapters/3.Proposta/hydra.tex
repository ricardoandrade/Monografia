\section{A aplicação Hydra}
\label{sec:hydra}
	
	
	Na visão da \textit{ubicomp}, o \textit{smart space} é composto por diversos dispositivos que
	fornecem uma gama de recursos e funcionalidades. É plausível de se esperar que muitos destes
	recursos sejam equivalentes entre si, como a existência de várias telas a disposição. Com tantas
	opções espera-se que um ambiente inteligente possibilite ao usuário escolher aquela que melhor lhe
	atende na tarefa sendo realizada. É neste cenário que a aplicação Hydra entra em cena. Fazendo uso
	da plataforma provida pelo \textit{middleware uOS}, esta aplicação permite que o usuário
	redirecionar os recursos de sua máquina para outros mais adequados no ambiente \cite{almeida}.

    Para melhor entender a atuação da Hydra no ambiente considere o seguinte exemplo de uso:
	 
	O ambiente é dotado dos seguintes dispositivos:
	 
	
	\begin{enumerate}
		\item Macbook e o \textit{notebook} HP disponibilizam os recursos de \textit{webcam}, saída de
		vídeo, mouse e teclado;
		\item \textit{Laptop} Dell disponibiliza somente os recursos de saída de vídeo, mouse e teclado;
		\item \textit{Smartphone} Galaxy SII disponibiliza os recursos de câmera, mouse e teclado;
		\item \textit{SmartTV} disponibiliza os recursos de saída de vídeo. 
	\end{enumerate}
	
	A seguir temos um exemplo de interação neste ambiente utilizando a Hydra:

	\begin{quote}
	
		\it 	
		Inicia-se uma reunião cujo propósito é alinhar os conhecimentos a respeito dos projetos do grupo 
		UnBiquitous, discutir as soluções propostas e apresentar os artigos base para os projetos. Os
		participantes da reunião são Marcelo (dono do notebook HP), Bruno (dono do smartphone Galaxy SII), Ricardo (dono do
		Macbook) e Fabrício (dono laptop Dell).
		
		Para facilitar a visualização dos tópicos a serem discutidos, Fabrício como líder do projeto,
		instalou a aplicação Hydra em seu laptop e redirecionou a tela do laptop para a SmartTV. Os
		demais dispositivos estão utilizando uma instância do middleware uOS, disponibilizando seus
		recursos ao ambiente.

		Fabrício mostra a pauta da reunião. Após a discussão dos primeiros tópicos, Ricardo sente a
		necessidade de complementar alguns tópicos apresentados. Ricardo então pede ao Fabrício que
		redirecione o recurso de teclado do laptop Dell para seu Macbook. Após acrescentar os novos
		tópicos, Ricardo pede ao Fabrício que libere o recurso do teclado, para que seja utilizado o
		teclado do laptop do Fabríco novamente. Bruno percebe a necessidade de apontar algumas questões
		nos novos tópicos criados pelo Ricardo e solicita ao Fabrício que o recurso de mouse seja
		utilizado em seu celular. Com os avanços dos apontamentos feitos pelo Bruno, Marcelo
		decide mostrar um vídeo que lhe chamou a atenção. Solicita então o uso do recurso de saída de vídeo para
		que possa exibi-lo na SmartTV. Após o término da exibição do vídeo o grupo encerra as discussões
		e finaliza a reunião.

	\end{quote}
		
	Ao final da reunião, a aplicação Hydra proporcionou uma visão diferente no ~\textit{laptop} Dell
	pertencente ao Fabrício devido ao redirecionamento dos recursos. Neste, os recursos de saída de
	vídeo e \textit{mouse} estão direcionados para outros dispositivos no ambiente. Desta forma o
	~\textit{laptop} do Fabrício pode ser visto como uma composição dos recursos que estão distribuídos
	no ambiente. 
	
	Através desse exemplo foi possível observar que os dispositivos foram redirecionados de uma
	forma manual, partindo de uma necessidade específica. No entanto, a Hydra pode implementar
	inteligências que simplifiquem essa seleção e o redirecionamento dos recursos.
	
~\subsection{Recursos}

	Com o propósito de preservar a invisibilidade do sistema, a interação do usuário com os recursos
	pode acontecer de três formas distintas:
	
	\begin{itemize}
	  \item Interação direta
	  
	  		Esta interação possibilita ao usuário escolher qual dispositivo será utilizado, levando
	  		em consideração as vantagens e desvantagens de cada recurso.
	  
	  \item Interação sugerida
	  
	  		O sistema apresenta, de forma organizada, as possibilidades disponíveis compatíveis com a
	  		necessidade do usuário em um determinado momento, sendo esta implementada através de um
	  		mecanismo de inteligência.
	  		
	  \item Interação automática
	  
	  		Nesta interação, o sistema leva em consideração a sensibilidade do contexto para selecionar,
	  		automaticamente, o dispositivo que mais adeque ao propósito de resolução da tarefa intencionada.
	  		
	\end{itemize}
	
	
	A Hydra implementa a Interação Sugerida seguindo os conceitos apresentados pela \textit{DSOA}.
	Ela oferece suporte para quatro tipos de recursos, sendo estes definidos no ambiente através de
	\textit{UpDriver's}:
	
	\begin{enumerate}
	  \item \textit{Mouse}
		
			Este \textit{driver} provê uma abstração para comandos enviados por um apontador. Os eventos
			gerados por esse \textit{driver}, como por exemplo um clique duplo, são enviados ao destinatário
			através de eventos compostos por mensagens. Estes eventos são reconhecidas na Hydra e a
			movimentação do ponteiro na Hydra é feita com o uso da classe \textit{Robot}.
			
			São exemplos de serviços disponibilizados por esse recurso: 
			\begin{itemize}
			  \item posicionamento do ponteiro;
			  \item clique simples;
			  \item duplo clique;
			  \item clique com o botão direito;
			  \item	\textit{scroll}. 
			\end{itemize}
		
	  		A implementação desse recurso está disponível para as versões JSE (\textit{Java Standard Edition)} 
	  		e JME (\textit{Java Micro Edition}).

	  \item Teclado
	  		
	  		Esse recurso possui um comportamento similar ao \textit{mouse}, no que diz respeito
	  		envio dos eventos capturados. Este por sua vez, implementa o reconhecimento de
	  		eventos para pressionamentos, simples e simultâneo, e liberação de teclas, realizando
	  		capturas para:
	  		
	  		\begin{itemize}
	  		  \item Caracteres, maiúsculos ou minúsculos;
	  		  \item Números, de 0 a 9;
	  		  \item Comandos de teclado (\textit{shift}, \textit{control} e \textit{alt});
	  		  \item Teclas de funções  (F1 a F12);
	  		  \item Botões do teclado numérico.
	  		\end{itemize}
	
	  		Devido a diversidade de leiautes  configuráveis pelo teclado, os eventos capturados podem ser
	  		interpretados pela Hydra de uma forma diferente. Um exemplo desse comportamento pode ser
	  		observado no evento de captura da tecla ``ç'', para o leiaute do teclado no padrão ABNT2.
	  		Neste caso, para o padrão ABNT, o evento que apresenta a tecla ``ç'' é configurado como uma
	  		combinação de eventos de outras teclas. Essas diferenças são esperadas e podem serem corrigidas
	  		através da configuração do novo leiaute.
	  		
	  		As mensagens recebidas pela Hydra são traduzidos em comandos, que por sua vez são emulados
	  		utilizando a classe \textit{Robot}.
	  		
	  \item Tela
	  	
	  		Esse recurso utiliza a classe \textit{Robot} para capturar as imagens da tela, ao qual
	  		posteriormente será sequencia e transformada em um formato de vídeo adequado para transmissão.
	  		A transmissão do ~\textit{streaming} de dados é feita através do protocolo RTP (\textit{Real
	  		Time Protocol}) utilizando o JMF (\textit{Java Media Framework}).

	  		Após o recebimento do conteúdo na Hydra, o \textit{streaming} de dados é interpretado, 
	  		reconstruído e posteriormente exibido na tela utilizada pela aplicação Hydra.
	  	
	  \item Câmera
	  
			De modo análogo a funcionalidade de tela, este \textit{driver} utiliza o protocolo RTP para
			realizar a transmissão do \textit{streaming} de dados. Adicionalmente, este \textit{driver}
			disponibiliza os serviços de seleção e ativação de câmeras presentes no dispositivo. Após a
			detecção das câmeras e da seleção de qual câmera utilizar, a aplicação Hydra receberá o
			\textit{streaming} de dados enviados pela câmera escolhida e exibirá o conteúdo recebido em uma
			nova janela na tela utilizada pela Hydra.
		
	\end{enumerate}
	
	Com o suporte para esses conjunto de recursos, a Hydra consegue abranger uma boa gama de
	cenários de uso. A maioria dos recursos estão classificados em um dos tipos de recursos
	suportados pela Hydra, possibilitando assim o redirecionamento de diversos recursos presentes no
	ambiente inteligente.
	
